\section{Introduction}

\subsection{Contexte}

Les données étudiées dans ce travail sont constituées d’images
ultrasonores acquises lors d’inspections de puits, représentant
des coupes radiales du tubage, du ciment et de la formation
environnante. Ces images, caractérisées par une forte variabilité
spatiale et un niveau de bruit significatif, contiennent une
information indirecte mais essentielle sur l’état du ciment, dont
la qualité conditionne l’intégrité mécanique et la sécurité du
puits. L’enjeu est d’exploiter ces données pour identifier
automatiquement, au niveau du pixel, des interfaces physiques
clés, en particulier la troisième interface d’écho entre le ciment
et la formation géologique. Le problème est formulé comme une
tâche de segmentation supervisée d’images, dans laquelle un modèle
d’apprentissage automatique doit apprendre à associer à chaque
pixel une classe correspondant à une structure physique donnée.
L’objectif de ce travail est ainsi de concevoir un pipeline de
vision par ordinateur permettant de traiter ces images, d’en
extraire des représentations pertinentes et de produire des
segmentations robustes et généralisables, évaluées à l’aide de
métriques standard de segmentation, afin de fournir une estimation
fiable et automatisée de la qualité du ciment à partir des données
ultrasonores.

\subsection{Equipe}

Nous sommes deux alternants en 3e année à l'école Centrale de Nantes,
en option Informatique pour l'intelligence artificielle